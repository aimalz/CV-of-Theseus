% this is a LaTeX file
%----------------------------------------------------------------------
% name:
%   malz_cv.tex
% purpose:
%   Malz's CV and list of publications
% to-do:
%   - Convert all \ads{} to \doi{} -- the DOIs should exist now.
%   - Is the ``what constrains'' paper published anywhere?
%   - Should I make an \acronym{} macro that uses \small?  I should.
% comments:
%   - Don't list full author lists of 10 or more.
%   - If you want to include/remove the date, see \renewcommand{\today}.
%   - Swap 1/0 flag to include invited talks.
%----------------------------------------------------------------------

\documentclass[11pt,letterpaper]{article}
\usepackage{fancyhdr, afterpage}
\usepackage{color, hyperref}
\usepackage{ifthen}
\usepackage[margin=1in]{geometry}
\usepackage{lastpage}
\usepackage{etaremune}

% put in the date -- or not
  \renewcommand{\today}%5{2019 May 22}

% hypertex insanity
  \definecolor{linkcolor}{rgb}{0,0,0.33}
  \definecolor{grey}{rgb}{0.5,0.5,0.5}
  \hypersetup{
    colorlinks=true,        % false: boxed links; true: colored links
    linkcolor=grey,        % color of internal links
    citecolor=linkcolor,    % color of links to bibliography
    filecolor=black,        % color of file links
    urlcolor=linkcolor      % color of external links
  }

  % repeated text formatting
  \newcommand{\current}{\deemph{current:}}
  \newcommand{\past}{\deemph{past:}}
  \newcommand{\latin}[1]{\textsl{#1}}
  \newcommand{\etal}{\latin{et~al.}}
  \newcommand{\satellite}[1]{\textsl{#1}}
  \newcommand{\project}[1]{\textsl{#1}}
  \newcommand{\usd}[1]{{#1}\,\textsc{usd}}
  \newcommand{\eur}[1]{{#1}\,\textsc{eur}}
  \newcommand{\grantnumber}[2]{{\footnotesize{#1}}; #2,~\textsc{pi}}
  % \newcommand{\talk}[4]{{\em #1}, #2, #3, #4}
  \newcommand{\talk}[4]{#3, #2}% \deemph{#4}}
  % \newcommand{\talksem}[3]{{\em #1}, #2, #3}
%  \newcommand{\talksem}[3]{#2 \deemph{#3}}
% literature links--use doi if you can
  \newcommand{\doi}[2]{\href{http://dx.doi.org/#1}{``{#2}''}}
  \newcommand{\ads}[2]{\href{http://adsabs.harvard.edu/abs/#1}{{#2}}}
  \newcommand{\isbn}[1]{{\footnotesize(\textsc{isbn:}{#1})}}
  \newcommand{\arxiv}[2]{\href{http://arxiv.org/abs/#1}{{#2}\ (\texttt{arXiv:#1})}}
% define de-emphasize commands
  \newcommand{\deemph}[1]{\textcolor{grey}{\footnotesize{#1}}}
%  \newcommand{\deemph}[1]{{\fontsize{9pt}{5pt}\selectfont\textcolor{grey}{#1}}}%\par}
  \newcommand{\pubnumber}[1]{\deemph{{#1}.}}
% define heading command SERIOUS HACKING HERE
  \newcommand{\malzheading}[1]{\addvspace{1ex}\pagebreak[3]%
    \par\textbf{#1}\nopagebreak%
    \afterpage{\markboth{Alex I. Malz, curriculum vitae continued: \textbf{#1}}{}}%
    \nopagebreak\\*[0.5ex]\nopagebreak}%
% custom list characteristics
  \newcounter{refpubnum}
  \newcommand{\malzlist}{%
    \rightmargin=0in
    \leftmargin=0.18in
    \topsep=0ex
    \partopsep=0pt
    \itemsep=0.2ex
    \parsep=0pt
    \itemindent=-1.0\leftmargin
    \listparindent=0.0\leftmargin
    \settowidth{\labelsep}{~}
    \usecounter{refpubnum}
  }
  \newcommand{\note}[1]{{\small\par{#1}}}
% Margins and spaces
  \raggedright
  \setlength{\oddsidemargin}{0in}
  \setlength{\topmargin}{-0.25in}
  \setlength{\headsep}{0.20in}
  \setlength{\headheight}{0.25in}
  \setlength{\textheight}{9.75in}
%  \addtolength{\textheight}{-\headsep}
%  \addtolength{\textheight}{-\headheight}
  \addtolength{\topmargin}{-\headsep}
  \addtolength{\topmargin}{-\headheight}
  \setlength{\textwidth}{6.50in}
  \setlength{\parindent}{0in}
  \setlength{\parskip}{1ex}
% Headings and footing
  \renewcommand{\headrulewidth}{0pt}
  \pagestyle{fancy}
  \lhead{\deemph{\leftmark}}
  \rhead{\deemph{\thepage}}
  \cfoot{}

%\newcommand{\descr}[1]{}
\newcommand{\descr}[1]{\\ \indent \deemph{#1}}

\rhead{\thepage\ of \pageref{LastPage}}
\begin{document}
%\thispagestyle{empty}
\sloppy\sloppypar\raggedbottom\frenchspacing
% \setcounter{page}{2}
\setlength{\tabcolsep}{0.55cm}
\begin{tabular}{lllll}
\noindent\textbf{\Large Alex I. Malz}            & & %{\small McWilliams Center for Cosmology} 
& & {\url{https://github.com/aimalz}} \\%[1ex]
\noindent ORCiD 0000-0002-8676-1622 & & %{\small Carnegie Mellon University (CMU)} 
& & \texttt{aimalz@nyu.edu}    
\end{tabular}\vspace{1ex} 

\malzheading{Professional Experience}
\begin{list}{}{\malzlist}
	\item Special Faculty, Carnegie Mellon University (CMU) McWilliams Center for Cosmology 2022--\\
	\deemph{\color{black} Project Scientist, LSST Interdisciplinary Network for Collaborative Computing (LINCC) Frameworks}
	\item Visiting Scientist, Lawrence Livermore National Laboratory (LLNL) 2021--
	\item Postdoctoral Research Fellow, German Centre for Cosmological Lensing (GCCL) 2019--22
\end{list}% HACK

\malzheading{Education \& Academic Awards}
\begin{list}{}{\malzlist}
\item
PhD 2020, Physics, New York University (NYU) Graduate School of Arts and Sciences (GSAS)
	\begin{list}{}{\malzlist}
		\item Thesis: \deemph{Probabilistic analysis methods for cosmology using uncertainty-dominated photometric data} \\
		Advisor: \deemph{David Hogg} DOI: \deemph{\hyperlink{https://zenodo.org/record/3973536}{dx.doi.org/10.5281/zenodo.3973536}}
%		\item Finalist, \href{https://youtu.be/vKs3PYqZWg8}{Dance Your Ph.D. Contest}, 2019
		\item GSAS Ted Keusseff Fellow 2018-19; GSAS MacCracken Fellow 2014-18
		\item
		Department of Energy Office of Science Graduate Student Research SCGSR Fellow 2017
	\end{list}{}
\item
MS 2014, Astronomy \& Astrophysics, Pennsylvania State University (PSU)
	\begin{list}{}{\malzlist}
		\item Astronomy \& Astrophysics Braddock/Roberts Fellow 2012-13
	\end{list}
\item
BS 2011, Physics \& History, California Institute of Technology (Caltech)
	\begin{list}{}{\malzlist}
		\item Caltech President's Scholar 2006-11
	\end{list}
\end{list}

%\malzheading{Awards \& Honors}% awards?
%\begin{list}{}{\malzlist}
%\item Finalist, Dance Your Ph.D. contest, 2019
%
%\end{list}

\malzheading{Leadership \& Scientific Collaborations}
\begin{list}{}{\malzlist}
	\item Cosmological Advanced Survey Telescope for Optical and ultraviolet Research (CASTOR), 2022--
	\item Vera Rubin Observatory Legacy Survey of Space and Time (LSST) Science Collaborations (SCs)
%	 Dark Energy (DE) and Informatics \& Statistics (IS) Science Collaborations (SCs) 
	\begin{list}{}{\malzlist}
		\item Dark Energy SC (DESC) Builder 2020--, Full Member 2016--, Member 2016--;
		\item DESC Photometric Redshifts Working Group Convener 2019-21; Topical Team Lead 2021--;
		\item DESC Collaboration Council 2018-20; Membership Committee 2017-19
		\item Informatics \& Statistics SC (ISSC) Member 2019--, Charter Committee 2023
		\item Galaxies, Transients \& Variable Stars, Active Galactic Nuclei SCs  Member 2022--
%		\item Transients \& Variable Stars (TVS) Science Collaboration Member 2022--
%		\item Active Galactic Nuclei (AGN) 
	\end{list}
	\item Kilo-Degree Survey (KiDS) Member, 2019--
	\item Cosmostatistics Initiative (COIN) Member 2018--
%	   \item Photometric LSST Astronomical Time-Series Classification Challenge (PLAsTiCC) Team Member, 2017--.
%\item NYU Center for Cosmology and Particle Physics, 2014--.
% \item American Astronomical Society (AAS) Junior Member, 2014--.
%\item PSU Institue for Gravitation and the Cosmos, 2012--2014
%	\item \past\ Hobby-Eberly Telescope Dark Energy Experiment (HETDEX) Member, 2012--2014
\end{list}

\malzheading{Grants}
 \begin{list}{}{\malzlist}
 	% Roman x2
 \item PI, DESC workshop 2023, Redshift Assessment Infrastructure Layers (RAIL) Topical Team
 \item PI, CMU McWilliams/PSC Seed Grant 2022-3 for ``Impossible Problems" seminar series
 \item PI, LSST Corporation (LSSTC) Enabling Science 2022 for ISSC Ambassador student travel
 \item LSSTC Enabling Science 2021
\begin{list}{}{\malzlist}
	\item Organizer, ISSC Ambassadors \deemph{6 of 8 applications for interdisciplinary student research funded}
	\item PI, Stress-testing multimodal photometric redshift posteriors in the extrapolative regime
\end{list}
 \item Co-I, LSSTC Enabling Science 2018 \& 2019, DESC Cluster Lensing Mass Modeling Sprint Week% at RUB
% \item Co-I, LSSTC Enabling Science 2018, DESC Cluster Lensing Mass Modeling Sprint Week% at CMU
\item Co-I, NSF 2015, AST-1517237: New Probabilistic Methods for Observational Cosmology
\end{list}
% begin HACK
% this next line is a clearpage line with heading hack. Feel free to delete it if pagination is okay.
%\afterpage{\markboth{Alex I. Malz, curriculum vitae continued: \textbf{\ref{rpcount} Refereed publications}}{}}%
%\clearpage
% end HACK

\malzheading{Research Supervision \& Mentoring}
\begin{list}{}{\malzlist}
	\item 2023--: Alice Crafford; CMU, BS \deemph{2025}
	\item 2021-22: Natalia Stylianou; U. of Leicester, BS \& MS \deemph{2022}; Oxford U., PhD. \deemph{2025}
	\item 2021-22: Nicola Hunfeld; Ruhr-U. Bochum, MS \deemph{2022}
	\item 2018: David Mykytyn, Dave Perrett, Ted Singer \& Zora Tung; non-academic professionals
	\item 2018: Marin Hyatt \& Lia Lubit; Hunter College High School \deemph{2020}
\end{list}

\afterpage{\markboth{Alex I. Malz, curriculum vitae continued: \textbf{Invited Talks}}{}}%
\clearpage

\malzheading{Invited Talks \& Panels}
% PCW 2022 ELAsTiCC and PZ sessions
% PCW 2021 PZ and ISSC sessions
\nopagebreak\begin{list}{}{\malzlist}
	\item 2023
	\nopagebreak\begin{list}{}{\malzlist}
		\item \talk{lincc frameworks}
		{3rd Nobel Turing Challenge Initiative Workshop}
		{CMU}
		{Summer 2023}
		\item \talk{Getting the most out of LSST with data-driven statistical methodologies to illuminate the dark universe}
		{Physics Colloquium}
		{Northeastern University}
		{Winter 2023}
		\item \talk{Challenges and Opportunities in the Rubin Observatory Ecosystem}
		{Statistical Methods for Physical Sciences (STAMPS) Colloquium}
		{CMU}
		{Winter 2023}
%		\item \talk{Tutorial: Photo-$z$ uncertainty}
%		{Building a Phys. Understanding of Galaxy Evol. with Data-driven Astro. Tutorial}
%		{KITP}
%		{Winter 2023}
	\end{list}
	\item 2022
	\nopagebreak\begin{list}{}{\malzlist}
		\item \talk{Challenges within challenges: classifying stars that go boom in the night}
		{Impossible Problems Interdisciplinary Wine-and-Cheese Seminar}
		{CMU}
		{Fall 2022}
		\item \talk{Evaluation of probabilistic photometric redshift estimation approaches for The Rubin Observatory Legacy Survey of Space and Time (LSST)}
		{Photometric Redshifts}
		{Pan-Survey SED Forum}
		{Winter 2022}
		\item \talk{How to get the most out of LSST: Principled methodologies for the uncertainty-dominated regime}
		{Astronomy Group Colloquium}
		{Stony Brook University}
		{Winter 2022}
		\item \talk{Novel statistical methods to probe the cosmic frontier with LSST}
		{Lab-wide Seminar}
		{Lawrence Livermore National Laboratory}
		{Winter 2022}
	\end{list}
	%\nopagebreak\begin{list}{}{\malzlist}
		\item 2021
		\nopagebreak\begin{list}{}{\malzlist}
			\item \talk{How, and how not, to obtain the redshift distribution from probabilistic photometric redshifts}
			{Origins Cluster Data Science Laboratory Seminar}
			{Ludwig-Maximilians-Universit{\"a}t Munich}% \& Technical University of Munich}
		{Summer 2021}
		\item \talk{Proceed with caution: how, and how not, to use machine learning to probe cosmology}
		{Machine Learning in Astro.}%, id. 103.02}
	{American Astronomical Society (AAS) \#238 Meeting-in-a-Meeting}
	{Spring 2021}
	\item \talk{The DESC Photometric Redshifts Working Group: Challenges and Opportunities}
	{Machine Learning Seminar}
	{Dark Energy Spectroscopic Instrument (DESI) Collaboration}
	{Spring 2021}
	\item \talk{How, and how not, to obtain the redshift distribution from probabilistic photometric redshifts}
	{Institute for Advanced Study Cosmology Lunch Talk}
	{Princeton University}
	{Spring 2021}
	\item \talk{Will Machine Learning Solve photometric redshifts? (No)}
	{Debate Panel: Will ML Solve Photometric Redshifts?}
	{Machine Learning Club}
	{Spring 2021}
\end{list}
\item 2020
\nopagebreak\begin{list}{}{\malzlist}
	\item \talk{Are we there yet? How to read the statistical signposts}
	{LSST-DESC Meeting, Dark Energy School Lesson}
	{University of Arizona}
	{Winter 2020}
\end{list}
\item 2019
\nopagebreak\begin{list}{}{\malzlist}
	\item \talk{How to Advance Cosmology with the Data Products of Machine Learning}
	{Accurate Lensing in the Era of Precision Cosmo. Workshop}
	{Berkeley Cen. for Cosm. Phys.}
	{Spring 2019}
	\item \talk{Principled Cosmology Inference in the Regime of Uncertainty-Dominated Data}
	{Institute for Astronomy Seminar}
	{Royal Observatory Edinburgh}
	{Spring 2019}
	\item \talk{Maximizing LSST Science with Probabilistic Data Products}
	{Data Intensive Research in Astronomy and Cosmology Seminar}
	{University of Washington}
	{Spring 2019}
\end{list}
\item 2018
\nopagebreak\begin{list}{}{\malzlist}
	\item \talk{Maximizing LSST Science with Probabilistic Data Products}
	{Cosmology Seminar}
	{Lawrence Livermore National Laboratory}
	{Fall 2018}
	\item \talk{Probabilistic Data Products for Next-Generation Ground-Based Photometric Surveys}
	{Berkeley Center for Cosmological Physics Seminar}
	{Lawrence Berkeley National Laboratory}
	{Fall 2018}
	\item \talk{Practicalities of Massive Next-Generation Photometric Galaxy Surveys}
	{Astronomy Seminar}
	{Leiden University}
	{Fall 2018}
	\item \talk{Challenges of working with photo-z PDFs}
	{Colours of the Universe Workshop}
	{University of Leiden Lorentz Center}
	{Fall 2018}
	\item \talk{Supernova Classification}
	{ML + Time Domain Workshop}
	{Carnegie Mellon University}
	{Summer 2018}
	\item \talk{How to Advance Cosmology with the Data Products of Machine Learning}
	{Seminar}
	{Laboratoire de Physique Corpusculaire de Clermont}
	{Summer 2018}
	\item \talk{Probabilistic Photometric Redshifts for Large Surveys}
	{Colloquium}
	{Laboratoire de Physique Subatomique et de Cosmologie de Grenoble}
	{Summer 2018}
\end{list}
\item 2017
\nopagebreak\begin{list}{}{\malzlist}
	\item \talk{Probabilistic Photometric Redshifts for Next-Generation Galaxy Surveys}
	{Astronomy Colloquium}
	{State University of New York at Stony Brook}
	{Fall 2017}
	\item \talk{Approximating Probabilistic Photometric Redshifts for Large Surveys}
	{Canadian Institute for Theoretical Astrophysics Seminar}
	{University of Toronto}
	{Fall 2017}
	\item \talk{Cosmological Hierarchical Inference with Probabilistic Photometric Redshifts}
	{Kavli Institute for Particle Astrophysics and Cosmology Tea}
	{Stanford University}
	{Summer 2017}
	\item \talk{Photo-z PDF Tests and Storage}
	{Photo-z Workshop for Large Surveys}
	{Tohoku University}
	{Spring 2017}
\end{list}
\item 2016
\nopagebreak\begin{list}{}{\malzlist}
	\item \talk{A Fully Probabilistic Approach to the Redshift Distribution Function}
	{Astronomy Seminar}
	{University College London}
	{Summer 2016}
	\item \talk{A Fully Probabilistic Approach to the Redshift Distribution Function}
	{Astronomy Seminar}
	{Imperial College London}
	{Summer 2016}
	\item \talk{A Fully Probabilistic Approach to the Redshift Distribution Function}
	{Photometric Redshifts for LSST Workshop}
	{University of Pittsburgh}
	{Spring 2016}
\end{list}
\end{list}

\afterpage{\markboth{Alex I. Malz, curriculum vitae continued: \textbf{Talks (continued), Service, Teaching \& Outreach}}{}}%
\clearpage

%% \afterpage{\markboth{Alex I. Malz, curriculum vitae continued: \textbf{Selected contributed talks}}{}}%
%% \clearpage
\malzheading{Selected Contributed Talks}% \& Hackathon Participation}
\begin{list}{}{\malzlist}
	\item 2023: \talk{resspect metrics}
	{Statistical Challenges in Modern Astronomy VIII}
	{PSU}
	{Summer 2023}
%	\item 2023: \talk{Scalable, time-dependent probabilistic light curve classifications}
%	{OzFink workshop}
%	{Swinburne University of Technology}
%	{Spring 2023}
%	\item 2023: \talk{Metrics of probabilistic classifications of the alert stream}
%	{The Extended PLAsTiCC (ELAsTiCC) pseudo-Special Session}
%	{AAS \#241}
%	{Spring 2023}
\item 2022: \talk{Are classification metrics good proxies for SN Ia cosmological constraining power?}
{Boom! A Workshop on Explosive Transients with LSST}
{U. Illinois Urbana-Champaign}
{Summer 2022}
%\item virtual Cluster Masses 2020/1 poster
%\item virtual SCMA VII poster
\item 2021: \talk{The DESC Photometric Redshifts Working Group: Challenges \& Opportunities}
{DESC Special Session}%, id. 443.05}
{AAS \#237}
{Spring 2020}
%\item 2020: Astro Hack Week (virtual)
%\item 2020: LSST-TVS/ISSC Hackathon (virtual)
\item 2020: \talk{The devil is in the details: interpreting probabilities from machine learning}
{Bayesian Deep Learning for Cosmology and Gravitational Waves}
{APC}%Astroparticle and Cosmology Laboratory, Université de Paris}
{Spring 2020}
%\item \talk{Experimental design and metrics for LSST-DESC}
%	{Tutorial, DESC Sprint Week}
%	{Texas A\&M University}
%	{Fall 2019}
\item 2019: \talk{Probabilistic data analysis methods for large photometric surveys}
{Surveys \& Large Programs (Dissertation presentation)}%, id. 313.05D}
{AAS \#233}
{Spring 2019}
%\item 2019: \talk{The Photometric LSST Astronomical Time Series Classification Challenge (PLAsTiCC): challenge design and evaluation criteria}
%{Larger Efforts in Education \& Public Outreach (PLAsTiCC)}%, id. 212.05}
%%	{American Astronomical Society
%{AAS \#233}
%{Spring 2019}
%\item 2018: Chamonix, France, COIN Residential Program \# 6
\item 2018: \talk{Toward stack-free cosmology}
%	{Chalkboard talk, 
{Statistical Challenges in Large-Scale Structure in the Era of LSST}
% % \afterpage{\markboth{}{}}%
{Oxford University}
{Spring 2018}
% ({\em chalkboard})
%\item 2018: Chania, Greece, COIN Residential Program \# 5
\item 2017: \talk{Supernova Cosmology Inference with Probabilistic Photometric Redshifts}
{Supernovae: The LSST Revolution}
%	{Center for Interdisciplinary Exploration and Research in Astrophysics, 
{Northwestern University}
{Spring 2017}
\item 2016: \talk{Inference of Galaxy Population Statistics Using Photometric Redshift Probability Distribution Functions}
{Statistical Challenges in Modern Astronomy VI}
{CMU}
{Spring 2016}
%\item 2016: Astro Hack Week, University of California Berkeley
\item 2016: \talk{Cosmological Inference Using Photometric Redshift Probability Distribution Functions}
{Statistical Challenges in 21st Century Cosmology}
{Chania, Greece}
{Spring 2016}
% \item 2015: Astro Hack Week, NYU
% \item 2014: \talk{Probing Gravity in the High-Redshift Universe with HETDEX}{Neighborhood Workshop on Astrophysics and Cosmology}{PSU}{Spring 2014}
% \item 2013: \talk{Lyman Alpha Emitters: Cosmic Auditors of Gravity'}{Lyman-Alpha Emitter Summit}{PSU}{Fall 2013}
\end{list}

%\afterpage{\markboth{Alex I. Malz, curriculum vitae continued: \textbf{Service, Teaching \& Outreach}}{}}%
%\clearpage
%

\malzheading{Professional Service}\nopagebreak\begin{list}{}{\malzlist}
%	\item WOULD FACILITATING PCW SESSIONS GO HERE?
	\item Tutorial Presenter, Building a Phys. Understanding of Galaxy Evol. with Data-driven Astro. \deemph{KITP-CCA Winter 2023}
	\item Teacher, LSSTC Data Science Fellowship Program \deemph{Northwestern U., Fall 2022}
	\item Hackathon Leader, From Quarks to Cosmos with AI \deemph{CMU, Summer 2021}
	\item Tutorial Co-organizer, Bayesian Deep Learning for Cosmology and Gravitational Waves \deemph{Astroparticle and Cosmology Laboratory (APC) Université de Paris, Winter 2020}
	\item Tutorial Organizer, DESC Sprint Week \deemph{Texas A\&M University, Fall 2019}
	\item Workshop Co-organizer, Photometric LSST Astronomical Time-Series Classification Challenge (PLAsTiCC) \deemph{NYU, Spring 2018}
	\item Hackathon Leader, Astro Hack Week \deemph{UC Berkeley, Fall 2017}
	\item Local Organizing Committee Member, Astro Hack Week \deemph{NYU, Fall 2015}
	\item Reviewer, NSF Astronomy and Astrophysics Research Grants (AAG) \deemph{years redacted for confidentiality}
\end{list}

\malzheading{Teaching}
\begin{list}{}{\malzlist}
\item Instructor on Record
\begin{list}{}{\malzlist}
	\item PSU ASTRO 011 Elementary Astronomy Laboratory (Spring 2013)
\end{list}
\item Teaching Assistant
\begin{list}{}{\malzlist}
	\item NYU PHYS-UA 7 The Universe: Its Nature and History (Spring 2018)
	\item NYU PHYS-UA 15 Introduction to Cosmology (Fall 2017)
	\item PSU ASTRO 120 The Big Bang Universe (Spring 2014)
	\item PSU ASTRO 292 Astronomy of the Distant Universe (Springs 2013, 2014)
	\item PSU ASTRO 291 Astronomical Methods and the Solar System (Falls 2012, 2013)
	\item PSU ASTRO 001 Astronomical Universe (Falls 2012, 2013)
\end{list}
\item Guest Lecturer
\begin{list}{}{\malzlist}
	\item PSU ASTRO 485 Introduction to High-Energy Astronomy (Fall 2013)%(10/2013)
	\item PSU ASTRO 291 Astronomical Methods and the Solar System (Fall 2012) %(12/2012)
	\item PSU ASTRO 001 Astronomical Universe (Fall 2012) %(10/2012)
\end{list}
\end{list}

\malzheading{Public Outreach}
\begin{list}{}{\malzlist}
	\item Judge, New York City Science and Engineering Fair finals event \deemph{Springs 2018, --21, --22}%(03/2021)
	\item Juror, German Young Physicists' Tournament North Rhine-Westphalia \deemph{Winter 2021}%(02/2021)
	\item Contestant, Dance Your Ph.D. \deemph{\href{https://youtu.be/vKs3PYqZWg8}{youtu.be/vKs3PYqZWg8} \deemph{Winter 2018-19}} %(01/2019)
	\begin{list}{}{\malzlist}
		\item \deemph{Produced, directed, choreographed, and danced an entertaining explanation of Malz \& Hogg (2021)}
	\end{list}
	\item Math Circle Speaker at Bridge to Enter Advanced Mathematics \deemph{\href{https://beammath.org}{beammath.org} \deemph{Summer 2018}}% (07/2018)
	\begin{list}{}{\malzlist}
		\item \deemph{Designed and taught applied geometry lessons in the context of astronomy}% to talented sixth graders at low-performing inner-city schools
	\end{list}
	%\item Judge, New York City Science and Engineering Fair finals event \deemph{Spring 2018, 2021}%(03/2018)
%	\item Guest Speaker, Hunter College High School Science Club \deemph{Spring 2016}%(04/2016)
	\item Judge, Pennsylvania Junior Academy of Science finals event \deemph{Spring 2013}%(05/2013)
	\item Outreach Developer \& Facilitator, PSU AstroFest \deemph{Summers 2012, --13, --14}
	\begin{list}{}{\malzlist}
		\item \deemph{Created and implemented \href{https://drive.google.com/drive/folders/1MZ_4X8ujOsBYnhq1Yu--n5ef3yLRmrej?usp=sharing}{tie-dye-based activities} teaching astronomy concepts}% for multiple age groups
	\end{list}
\end{list}


%\malzheading{Published Software} 
%{\fontsize{9pt}{5pt}\selectfont\textcolor{grey}{AIM's code, including work in progress, is publicly available on GitHub; these have formal releases with DOIs.}\par}
%% at \hyperlink{https://github.com/aimalz}{github.com/aimalz} or \hyperlink{https://github.com/LSSTDESC}{github.com/LSSTDESC}.}\par}
%% add in licenses, summary of software, URL
%\nopagebreak\begin{list}{}{\malzlist}
%
%% qit%Conceptualization
%%Data curation
%%Formal analysis
%%Funding acquisition
%%Investigation
%%Methodology
%%Project administration
%%Resources
%%Software
%%Supervision
%%Validation
%%Visualization
%%Writing - original draft.
%%Writing - review & editing
%
%% recidivator/pollock
%% RAIL, qp
%
%\item LSST-DESC Redshift Assessment Infrastructure Layers Team (led by {\bfseries{A.I. Malz}}). 2022. \href{https://github.com/LSSTDESC/qp}{``{\tt qp}"}
%\descr{{\bf Lead author}: conceptualization, methodology, project administration, software, supervision, validation, writing (documentation)}
%
%\item LSST-DESC Redshift Assessment Infrastructure Layers Team (led by {\bfseries{A.I. Malz}}). 2022. \doi{10.5281/zenodo.7076131}{\tt RAIL}.
%\descr{{\bf Lead author}: conceptualization, methodology, project administration, software, supervision, validation, writing (documentation)}
%
%\item LSST-DESC CLMassMod Team (led by {\bf A.I. Malz}). 2021. \doi{10.5281/zenodo.5596167}{\tt CLMM}.
%\descr{{\bf Lead author}: conceptualization, funding acquisition, methodology, project administration, software, supervision, writing (documentation)}
%
%\item {\bf A.I. Malz}. 2020. \doi{10.5281/zenodo.4085252}{\tt chippr}. \deemph{\bf Sole author} %\deemph{(Cosmological Hierarchical Inference with Probabilistic Photometric Redshifts: \textit{implementation of the CHIPPR method of Malz \& Hogg 2021})} 
%
%\item {\bf A.I. Malz}, et al. 2019. \doi{10.5281/zenodo.3352639}{\tt ProClaM}. %\deemph{(Probabilistic Classification Metrics: \textit{library for evaluating metrics of probabilistic classifications})}
%\descr{{\bf Lead author}: conceptualization, methodology, software, validation, visualization, writing (documentation)}
%
%\item B. Brewer, T.K. Leung \& {\bf A.I. Malz}. 2018. \doi{10.5281/zenodo.1410782}{\tt StarStudded}. %\deemph{(\textit{package for generating probabilistic catalogs of crowded stellar fields})}
%\descr{Contributor: software}
%
%\item {\bf A.I. Malz} \& P.J. Marshall. 2017. \doi{10.5281/zenodo.1133465}{\tt qp}. %\deemph{(Quantile Parameterization: \textit{toolkit for handling diverse parameterizations of univariate probability density functions})}
%\descr{{\bf Lead author}: conceptualization, methodology, software, validation, visualization, writing (documentation)}
%\end{list}
%% \item
%% Foreman-Mackey,~D., Hogg,~D.~W., Lang,~D., \& Goodman,~J., 2012,
%% {\project{emcee} codebase}, MIT License,
%% an adaptive ensemble sampler
%% (\url{http://danfm.ca/emcee/}).


\afterpage{\markboth{Alex I. Malz, curriculum vitae continued: \textbf{Publications}}{}}%
\clearpage
 % \afterpage{\markboth{}{}}%
%
%
%\clearpage

\hypersetup{linkcolor=black}%
%\malzheading{ Refereed publications}%% use \etal when there are >9 authors!

%Conceptualization
%Data curation
%Formal analysis
%Funding acquisition
%Investigation
%Methodology
%Project administration
%Resources
%Software
%Supervision
%Validation
%Visualization
%Writing - original draft.
%Writing - review & editing

\malzheading{Publications}
\hypersetup{linkcolor=grey}
%\deemph{19 completed peer review; 10 as (co-)lead author; H-index=11}

%{\setlength{\baselineskip}{1ex}
{\fontsize{9pt}{5pt}\selectfont\textcolor{grey}{AIM has been awarded Builder Status within the LSST-DESC, which grants authorship rights on all DESC publications enabled by his work on photometric redshifts, time-domain classification, and service to the collaboration; 
however, he has chosen to only co-author papers to which he made \textit{a direct, scientific contribution}.}\par}
%}\par
%\setlength{\baselineskip}{1em}
%\textcolor{black}
\begin{etaremune}
%\begin{list}{\pubnumber{\therefpubnum}}{\malzlist}
%\item RAIL v1
%\item pzflow paper
%\item ELAsTiCC metrics
%\item ELAsTiCC release note
%\item CASTORpz
%\item lost and found
%\item 
%\item recidivator/pollock?
%\item SCIPPR

\item {\bf A.I. Malz}, F. Lanusse, J.F. Crenshaw, M.L. Graham. 2021. {\em submitted to ApJS 30 May 2023.} \arxiv{2104.08229}{An information-based metric for observing strategy optimization, demonstrated in the context of photometric redshifts with applications to cosmology}
\descr{{\color{black} \bf Lead author}: conceptualization, data curation, formal analysis, funding acquisition, investigation, methodology, software, validation, visualization, writing}% (original draft)}% \& review/editing)}

\item {\bf A.I. Malz}, M. Dai, K.A. Ponder, E.E.O. Ishida, S. Gonzalez-Gaitain, R. Durgesh, A. Krone-Martins, R.S. de Souza, N. Kennamer, S. Sreejith, L. Galbany. 2023. {\em submitted to A\&A 12 May 2023.} \arxiv{2305.14421}{Are classification metrics good proxies for SN Ia cosmological constraining power?}
\descr{{\color{black} \bf Lead author}: conceptualization, formal analysis, investigation, methodology, software, validation, visualization, writing}% - original draft, writing - review \& editing)}

\item A. Gagliano, G. Contardo, D. Foreman-Mackey, {\bf A.I. Malz}, P.D. Aleo. 2022. {\em accepted to ApJ 15 May 2023.} \arxiv{2305.08894}{First Impressions: Early-Time Classification of Supernovae using Host Galaxy Information and Shallow Learning}.
\descr{{\bf Contributor}: data curation, resources, software, writing}% (review/editing)}
	
\item R. Hlo{\v z}ek, {\bf A.I. Malz}, K.A. Ponder, M. Dai, G. Narayan, E.E.O. Ishida,% et al.
 T. Allam Jr., A. Bahmanyar, R. Biswas, L. Galbany, S.W. Jha, D.O. Jones, R. Kessler, M. Lochner, A.A. Mahabal, K.S. Mandel, J.R. Martinez-Galarza, J.D. McEwen, D. Muthukrishna, H.V. Peiris, C.M. Peters, C.N. Setzer. 2020. {\em accepted to ApJS 11 May 2023.} 
\arxiv{2012.12392}{Results of the Photometric LSST Astronomical Time-series Classification Challenge (PLAsTiCC)}
	\descr{{\color{black} \bf Lead author}: conceptualization, formal analysis, investigation, methodology, visualization, writing}% (original draft)}% \& review/editing)}

\item  M. Lokken, A. Gagliano, G. Narayan, R. Hlo{\v z}ek, R. Kessler, J.F. Crenshaw, L. Salo, C.S. Alves, D. Chatterjee, M. Vincenzi, {\bf A.I. Malz}. 2023. MNRAS 520 2. \doi{10.1093/mnras/stad302}{The Simulated Catalogue of Optical Transients and Correlated Hosts (SCOTCH)}
\descr{{\bf Contributor}: conceptualization, data curation, methodology}% (original draft)}% \& review/editing)}

\item N. Stylianou, {\bf A.I. Malz}, P. Hatfield, J.F. Crenshaw, J. Gschwend. 2021. PASP 134 1034. \doi{10.1088/1538-3873/ac59bf}{The sensitivity of GPz estimates of photo-z posterior PDFs to realistically complex training set imperfections}
\descr{{\color{black} \bf Lead author}: conceptualization, data curation, formal analysis, funding acquisition, investigation, methodology, software, supervision, validation, visualization, writing}% (original draft)}% \& review/editing)}

\item {\bf A.I. Malz}, D.W. Hogg. 2021. ApJ 928 127. \doi{10.3847/1538-4357/ac062f}{How to obtain the redshift distribution from probabilistic redshift estimates}
	\descr{{\color{black} \bf Lead author}: conceptualization, data curation, formal analysis, funding acquisition, investigation, methodology, software, validation, visualization, writing}% (original draft)}% \& review/editing)}
	
\item M. Aguena, C. Avestruz, C. Combet, S. Fu, R. Herbonnet, {\bf A. I. Malz}, M. Penna-Lima, M. Ricci, S. D. P. Vitenti, %et al.
L. Baumont, H. Fan, M. Fong, M. Ho, M. Kirby, C. Payerne, D. Boutigny, B. Lee, B. Liu, T. McClintock, H. Miyatake, C. Sifón, A. von der Linden, H. Wu, M. Yoon. 
2021. MNRAS 508 4 6092. \doi{10.1093/mnras/stab2764}{CLMM: a LSST-DESC Cluster weak Lensing Mass Modeling library for cosmology}
	\descr{{\color{black} \bf Lead author}: conceptualization, funding acquisition, methodology, project administration, software, validation, writing}% (original draft)}% \& review/editing)}
	
\item B. Dey, J.A. Newman, B.H. Andrews, R. Izbicki, A.B. Lee, D. Zhao, M.M. Rau, {\bf A.I. Malz}. 
2021. Fourth Workshop on Machine Learning and the Physical Sciences NeurIPS. \arxiv{2110.15209}{Re-calibrating Photometric Redshift Probability Distributions Using Feature-space Regression}
\descr{{\bf Contributor}: conceptualization, methodology, supervision, writing}
%\item B. Dey, B. Andrews, J.A. Newman, A.B. Lee, D. Zhao, R. Izbicki, \textbf{A.I. Malz}, A. Campos, M. Stanley. 2021. {\em submitted to NeurIPS Machine Learning and the Physical Sciences.} {Re-calibrating Probability Estimates using Local Coverage}
%	\descr{Contributor: conceptualization, data curation, investigation, methodology, software, supervision}

\item J. Zuntz, F. Lanusse, \textbf{A.I. Malz}, A.H. Wright, A. Slosar, et al. 2021. OJA 4. \doi{10.21105/astro.2108.13418} {The LSST-DESC 3x2pt Tomography Optimization Challenge}
\descr{{\color{black} \bf Lead author}: conceptualization, formal analysis, methodology, visualization, writing}% (original draft)}% \& review/editing)}

\item {\bf A.I. Malz}. 2020. PRD 103 083502. \doi{10.1103/PhysRevD.103.083502}{How NOT to obtain the redshift distribution from probabilistic redshift estimates}
	\descr{{\color{black} \bf Sole author}}
	
%	\item LSST-DESC, et al. (incl. {\bf A.I. Malz}) 2020. {\em DESC Note.}  \doi{10.5281/zenodo.4004760}{The LSST-DESC Science Roadmap}
%		\descr{Contributor: conceptualization, project administration, writing}% (original draft)}% \& review/editing)}

\item B. Moews, M.S. Schmitz, A.J. Lawler, J. Zuntz, {\bf A.I. Malz}, R.S. de Souza, R. Vilalta, A. Krone-Martins, E.E.O. Ishida. 2020. MNRAS 500 1 859. \doi{10.1093/mnras/staa3204}{Ridges in the Dark Energy Survey for cosmic trough identification}
	\descr{{\bf Contributor}: conceptualization, methodology, writing}% (original draft)}% \& review/editing)}
  
\item S.J. Schmidt, {\bf A.I. Malz}, J.Y.H. Soo, %et al.
I.A. Almosallam, M. Brescia, S. Cavuoti, J. Johen-Tanugi, A.J. Connolly, J. DeRose, P.E. Freeman, M.L. Graham, K.G. Iyer, M.J. Jarvis, J.B. Kalmbach, E. Kovacs, A.B. Lee, G. Longo, C.B. Morrison, J.A. Newman, E. Nourbakhsh, E. Nuss, T. Pospisil, H. Tranin, R.H. Wechsler, R. Zhou, R. Izbicki. 
2020. MNRAS 499 2 1587. \doi{10.1093/mnras/staa2799}{Evaluation of probabilistic photometric redshift estimation approaches for LSST}
  \descr{{\color{black} \bf Lead author}: conceptualization, formal analysis, investigation, methodology, project administration, software, supervision, validation, visualization, writing}% (original draft)}% \& review/editing)}

\item N. Kennamer, E.E.O. Ishida, S. Gonzalez-Gaitan, R.S. de Souza, A. Ihler, K. Ponder, R. Vilalta, A. Moller, D.O. Jones, M. Dai, A. Krone-Martins, B. Quint, S. Sreejitch, {\bf A.I.Malz}, L. Galbany. 2020. IEEE Symposium Series on
Computational Intelligence. \arxiv{2010.05941}{Active learning with RESSPECT: Resource allocation for extragalactic astronomical transients}
  \descr{{\bf Contributor}: conceptualization, methodology}

\item N. Dalmasso, T. Pospisil, A.B. Lee, R. Izbicki, P.E. Freeman, {\bf A.I. Malz}. 2019. As. \& Com. 20 100362. \doi{10.1016/j.ascom.2019.100362}{Conditional Density Estimation Tools in Python and R with Applications to Photometric Redshifts and Likelihood-Free Cosmological Inference}
  \descr{{\bf Contributor}: data curation, writing}% (original draft)}% \& review/editing)}

\item B. Moews, R.S. de Souza, E.E.O. Ishida, {\bf A.I. Malz}, C. Heneka, R. Vilalta, J. Zuntz. 2019. PRD 99 123529. \doi{10.1103/PhysRevD.99.123529}{Stress testing the dark energy equation of state imprint on supernova data}
  \descr{{\bf Contributor}: conceptualization, formal analysis, investigation, methodology, validation, writing}% (original draft)}% \& review/editing)}

\item T. Cantat-Gaudin, A. Krone-Martins, N. Sedaghat, A. Farahi, R.S. de Souza, R. Skalidis, {\bf A.I. Malz}, S. Macedo, B. Moews, C. Jordi, A. Moitinho, A. Castro-Ginard, E.E.O. Ishida, C. Heneka, A. Boucaud, A.M.M. Trindade. 2019. A\&A 624 A126. \doi{10.1051/0004-6361/201834453}{Gaia DR2 unravels incompleteness of nearby cluster population: New open clusters in the direction of Perseus}
  \descr{{\bf Contributor}: conceptualization, writing}% (original draft)}% \& review/editing)}

\item {\bf A.I. Malz}, R. Hlo{\v z}ek, %et al.
T. Allam Jr., A. Bahmanyar, R. Biswas, M. Dai, L. Galbany, E.E.O. Ishida, S.W. Jha, D.O. Jones, R. Kessler, M. Lochner, A.A. Mahabal, K.S. Mandel, J.R. Martinez-Galarza, J.D. McEwen, D. Muthukrishna, G. Narayan, H.V. Peiris, C.M. Peters, K. Ponder, C.N. Setzer. 
2019. AJ 158 5 171. \doi{10.3847/1538-3881/ab3a2f}{The Photometric LSST Astronomical Time-series Classification Challenge (PLAsTiCC): Selection of a performance metric for classification probabilities balancing diverse science goals}
  \descr{{\color{black} \bf Lead author}: conceptualization, data curation, formal analysis, investigation, methodology, project administration, software, supervision, validation, visualization, writing}% (original draft)}% \& review/editing)}

%	\item T. Allam Jr., A. Bahmanyar, R. Biswas, M. Dai, L. Galbany, R. Hlo{\v z}ek, E.E.O. Ishida, S.W. Jha, D.O. Jones, R. Kessler, M. Lochner, A.A. Mahabal, {\bf A.I. Malz}, K.S. Mandel, J.R. Martinez-Galarza, J.D. McEwen, D. Muthukrishna, G. Narayan, H.V. Peiris, C.M. Peters, K. Ponder, C.N. Setzer. {\em LSST-DESC Research Note.} 2018. \arxiv{1810.00001}{The Photometric LSST Astronomical Time-series Classification Challenge (PLAsTiCC): Data set}
%  		\descr{Contributor: conceptualization, methodology, software}%, writing (review/editing)}

\item C. Chang, M. Wang, S. Dodelson, T. Eifler, C. Heymans, M. Jarvis, M.J. Jee, S. Joudaki, E. Krause, {\bf A.I. Malz}, R. Mandelbaum, I. Mohammed, M. Schneider, M. Simet, M.A. Troxel, J. Zuntz. 2018. MNRAS 482 3 3696. \doi{10.1093/mnras/sty2902}{A Unified Analysis of Four Cosmic Shear Surveys}
  \descr{{\bf Contributor}: methodology, writing}% (original draft)}% \& review/editing)}

\item {\bf A.I. Malz}, P.J. Marshall, %et al.
M.L. Graham, S.J. Schmidt, J. DeRose, R. Wechsler. 
2018. AJ 156 0 35. \doi{10.3847/1538-3881/aac6b5}{Approximating photo-z PDFs for large surveys}
  \descr{{\color{black} \bf Lead author}: conceptualization, data curation, formal analysis, funding acquisition, investigation, methodology, software, validation, visualization, writing}% (original draft)}% \& review/editing)}
  
%  	\item P.J. Marshall, et al. (incl. {\bf A.I. Malz}) 2017. {\em whitepaper.} \arxiv{1708.04058}{Science-Driven Optimization of the LSST Observing Strategy}
%  	\descr{Contributor: conceptualization, methodology, writing}% (original draft)}% \& review/editing)}

\item A.S. Leung, V. Acquaviva, E. Gawiser, R. Ciardullo, E. Komatsu, {\bf A.I. Malz}, G.R. Zeimann, %et.al.
J.S. Bridge, N. Drory, J.J Feldmeier, S.L. Finkelstein, K. Gebhardt, C. Gronwall, A. Hagen, G.J. Hill, D.P. Schneider. 
2017. ApJ 843 2 130. \doi{10.3847/1538-4357/aa71af}{Bayesian Redshift Classification of Emission-Line Galaxies with Photometric Equivalent Widths}
  \descr{{\bf Contributor}: conceptualization}%, writing (review/editing)}

\item J.S. Bridge, C. Gronwall, R. Ciardullo, A. Hagen, G. Zeimann, {\bf A.I. Malz}, V. Acquaviva, D.P. Schneider, N. Drory, K. Gebhardt, S. Jogee. 2015. ApJ 799 2 205. \doi{10.1088/0004-637X/799/2/205}{Physical and Morphological Properties of [O II] Emitting Galaxies in the HETDEX Pilot Survey}
  \descr{{\bf Contributor}: conceptualization, methodology}%, writing (review/editing)}

\item R. Ciardullo, G.R. Zeimann, C. Gronwall, H. Gebhardt, D.P. Schneider, A. Hagen, {\bf A.I. Malz}, G.A. Blanc, G.J. Hill, N. Drory, E. Gawiser. 2014. ApJ 796 1 64. \doi{10.1088/0004-637X/796/1/64}{HST Emission Line Galaxies at z \~ 2: The Ly-alpha Escape Fraction}
  \descr{{\bf Contributor}: conceptualization, methodology}%, writing (review/editing)}

\item A. Hagen, R. Ciardullo, C. Gronwall, V. Acquaviva, J. Bridge, G.R. Zeimann, G.A. Blanc, N.A. Bond, S.L. Finkelstein, M. Song, E. Gawiser, D.B. Fox, H. Gebhardt, {\bf A.I. Malz}, D.P. Schneider, N. Drory, K. Gebhardt, G.J. Hill. 2014. ApJ 786 1 59. \doi{10.1088/0004-637X/786/1/59}{Spectral Energy Distribution Fitting of HETDEX Pilot Survey Lyman-alpha Emitters in COSMOS and GOODS-N}
  \descr{{\bf Contributor}: conceptualization, methodology}%, writing (review/editing)}
  
%\end{list}
\end{etaremune}


%\clearpage

%\malzheading{Unrefereed publications}%
%\nopagebreak\begin{list}{}{\malzlist}
	%\end{list}
\malzheading{Published Software} 
	{\fontsize{9pt}{5pt}\selectfont\textcolor{grey}{All of AIM's code, including work in progress, is publicly available on GitHub; these have formal releases with DOIs.}\par}
	% at \hyperlink{https://github.com/aimalz}{github.com/aimalz} or \hyperlink{https://github.com/LSSTDESC}{github.com/LSSTDESC}.}\par}
% add in licenses, summary of software, URL
\nopagebreak\begin{list}{}{\malzlist}

%Conceptualization
%Data curation
%Formal analysis
%Funding acquisition
%Investigation
%Methodology
%Project administration
%Resources
%Software
%Supervision
%Validation
%Visualization
%Writing - original draft.
%Writing - review & editing

% recidivator/pollock
% ReSSpect

\item LSST-DESC RAIL Topical Team (led by {\bfseries{A.I. Malz}}). 2022. \href{https://github.com/LSSTDESC/qp}{``{\tt qp}"}
\descr{{\bf Lead author}: conceptualization, methodology, project administration, software, supervision, validation, writing (documentation)}

\item LSST-DESC RAIL Topical Team (led by {\bfseries{A.I. Malz}}). 2022. \doi{10.5281/zenodo.7076131}{\tt RAIL}.
\descr{{\bf Lead author}: conceptualization, methodology, project administration, software, supervision, validation, writing (documentation)}

\item LSST-DESC CLMassMod Team (led by {\bf A.I. Malz}). 2021. \doi{10.5281/zenodo.5596167}{\tt CLMM}.
\descr{{\bf Lead author}: conceptualization, funding acquisition, methodology, project administration, software, supervision, writing (documentation)}

\item {\bf A.I. Malz}. 2020. \doi{10.5281/zenodo.4085252}{\tt chippr}. %\deemph{(Cosmological Hierarchical Inference with Probabilistic Photometric Redshifts: \textit{implementation of the CHIPPR method of Malz \& Hogg 2021})}
\descr{\bf Sole author}

\item {\bf A.I. Malz}, et al. 2019. \doi{10.5281/zenodo.3352639}{\tt ProClaM}. %\deemph{(Probabilistic Classification Metrics: \textit{library for evaluating metrics of probabilistic classifications})}
\descr{{\bf Lead author}: conceptualization, methodology, software, validation, visualization, writing (documentation)}

\item B. Brewer, T.K. Leung \& {\bf A.I. Malz}. 2018. \doi{10.5281/zenodo.1410782}{\tt StarStudded}. %\deemph{(\textit{package for generating probabilistic catalogs of crowded stellar fields})}
\descr{Contributor: software}

\item {\bf A.I. Malz} \& P.J. Marshall. 2017. \doi{10.5281/zenodo.1133465}{\tt qp}. %\deemph{(Quantile Parameterization: \textit{toolkit for handling diverse parameterizations of univariate probability density functions})}
\descr{{\bf Lead author}: conceptualization, methodology, software, validation, visualization, writing (documentation)}
\end{list}
% \item
% Foreman-Mackey,~D., Hogg,~D.~W., Lang,~D., \& Goodman,~J., 2012,
% {\project{emcee} codebase}, MIT License,
% an adaptive ensemble sampler
% (\url{http://danfm.ca/emcee/}).

%\clearpage

\malzheading{Non-standard Publications}
% add in licenses, summary of software, URL
\nopagebreak\begin{list}{}{\malzlist}
	
	\item K. Breivik, et al. (incl. {\bf A.I. Malz}) 2022. {\em whitepaper.} \arxiv{2208.02781}{From Data to Software to Science with the Rubin Observatory LSST }
		\descr{Contributor: conceptualization, investigation, writing}% (original draft)}% \& review/editing)}
	
	\item LSST-DESC, et al. (incl. {\bf A.I. Malz}) 2020. {\em LSST-DESC Research Note.}  \doi{10.5281/zenodo.5527255}{The LSST-DESC Science Roadmap}% https://doi.org/10.5281/zenodo.3547566\doi{10.5281/zenodo.4004760}
		\descr{Contributor: conceptualization, project administration, writing}% (original draft)}% \& review/editing)}
	
	\item T. Allam Jr., A. Bahmanyar, R. Biswas, M. Dai, L. Galbany, R. Hlo{\v z}ek, E.E.O. Ishida, S.W. Jha, D.O. Jones, R. Kessler, M. Lochner, A.A. Mahabal, {\bf A.I. Malz}, K.S. Mandel, J.R. Martinez-Galarza, J.D. McEwen, D. Muthukrishna, G. Narayan, H.V. Peiris, C.M. Peters, K. Ponder, C.N. Setzer. {\em LSST-DESC Research Note.} 2018. \arxiv{1810.00001}{The Photometric LSST Astronomical Time-series Classification Challenge (PLAsTiCC): Data set}
	\descr{Contributor: conceptualization, methodology, software}%, writing (review/editing)}
	
	\item {\bf A.I. Malz}, et al. 2018. \doi{10.5281/zenodo.6382752}{Dance Your Ph.D. 2018/9: Probabilistic methods for cosmological analysis with uncertainty-dominated data} 
	\deemph{\href{https://youtu.be/vKs3PYqZWg8}{(educational music video)}}
		\descr{{\bf Lead author}: conceptualization, funding acquisition, methodology (choreography), project administration (production), resources (costumes), software (video editing \& web maintenance), supervision, visualization}
	
	\item P.J. Marshall, et al. (incl. {\bf A.I. Malz}) 2017. {\em whitepaper.}  \arxiv{1708.04058}{Science-Driven Optimization of the LSST Observing Strategy}
		\descr{Contributor: conceptualization, methodology, writing}% (original draft)}% \& review/editing)}
	
	\item {\bf A.I. Malz}. 2017. Cooper Square Review. \href{http://coopersquarereview.org/post/going-nowhere-fast/}{``Going nowhere fast''} \deemph{(science communication essay)}
		\descr{\bf Sole author}
\end{list}


%\clearpage
\malzheading{Citeable Presentations}
% add in licenses, summary of software, URL
\nopagebreak\begin{list}{}{\malzlist}
	
% other LINCC AAS 241 posters!

	\item {\bf A.I. Malz} \& the Extended PLAsTiCC (ELAsTiCC) Team. 2023. American Astronomical Society, AAS Meeting \#241, id. 117.04 ``ELAsTiCC: Metrics of probabilistic classifications of the alert stream" \deemph{(contributed talk)}
	%	\descr{{\bf Lead author}: conceptualization, formal analysis, investigation, methodology, project administration, software, validation, visualization, writing}
	
	\item {\bf A.I. Malz} \& the LSST-DESC RAIL Team. 2023. American Astronomical Society, AAS Meeting \#241, Astronomy and Cloud Computing Special Session, id. 358.01. ``All aboard! A LINCC Framework for extragalactic science using RAIL" \deemph{(contributed poster)}
	%	\descr{{\bf Lead author}: conceptualization, formal analysis, investigation, methodology, project administration, software, validation, visualization, writing}
	
	\item {\bf A.I. Malz}. 2021. American Astronomical Society, AAS Meeting \#238, Machine Learning in Astronomy Meeting-in-a-Meeting, id. 103.02. ``Proceed with caution: how, and how not, to use machine learning to probe cosmology" \deemph{(invited talk)}
	%	\descr{{\bf Sole author}}
	
	\item {\bf A.I. Malz}, F. Lanusse, J.F. Crenshaw, M.L. Graham. 2021. American Astronomical Society, AAS Meeting \#238, id. 230.04. ``\texttt{TheLastMetric}: an information-based observing strategy metric for photometric redshifts, cosmology, and more" \deemph{(contributed poster)}
%	\descr{{\bf Lead author}: conceptualization, formal analysis, investigation, methodology, project administration, software, validation, visualization, writing}
%	 (original draft)}% \& review/editing)}
	
	\item J.F. Crenshaw, J.B. Kalmbach, {\bf A.I. Malz}, A.J. Connolly. 2021. American Astronomical Society, AAS Meeting \#238, id. 230.01. ``\texttt{PZFlow}: normalizing flows for cosmology, with applications to forward modeling galaxy photometry" \deemph{(contributed poster)}
%	\descr{Contributor: supervision, validation}
	
	\item {\bf A.I. Malz}. 2021. American Astronomical Society, AAS Meeting \#237, LSST-DESC Special Session, id. 443.05. ``The DESC Photometric Redshifts Working Group: Challenges \& Opportunities'' \deemph{(contributed talk)}
%	\descr{{\bf Sole author}}
	
	\item {\bf A.I. Malz}. 2019. American Astronomical Society, AAS Meeting \#233, Surveys \& Large Programs, id. 313.05D. ``Probabilistic data analysis methods for large photometric surveys'' \deemph{(contributed talk)}
%	\descr{{\bf Sole author}}
	%
	\item {\bf A.I. Malz}. 2019. American Astronomical Society, AAS Meeting \#233, Larger Efforts in Education \& Public Outreach, id. 212.05. ``The Photometric LSST Astronomical Time Series Classification Challenge (PLAsTiCC): challenge design and evaluation criteria” \deemph{(contributed talk)}
%	\descr{{\bf Sole author}}
	
	\item C.M. Peters, {\bf A.I. Malz} \& R. Hlozek. 2018. American Astronomical Society, AAS Meeting \#231, id. 245.03. ``Supernova Cosmology Inference with Probabilistic Photometric Redshifts'' \deemph{(contributed poster)}
%	\descr{{\bf Lead author}: conceptualization, data curation, formal analysis, investigation, methodology, software, validation, visualization, writing}
%	 (original draft)}% \& review/editing)}
	
	\item {\bf A.I. Malz} \& S. Shandera. 2014. American Astronomical Society, AAS Meeting \#223, id. 456.04. ``Probing Gravity in the High-Redshift Universe with HETDEX'' \deemph{(contributed poster)}
%	\descr{{\bf Lead author}; conceptualization, data curation, formal analysis, investigation, methodology, visualization, writing}
%	 (original draft)}% \& review/editing)}
	
	\item {\bf A.I. Malz}, R. Rich \& S. Lepine. 2009. American Astronomical Society, AAS Meeting \#213, id. 602.04. ``Low-mass Binaries in the Galactic Halo Resolved by Adaptive Optics'' \deemph{(contributed poster)}
%	\descr{{\bf Lead author}; formal analysis, funding acquisition, investigation, writing}
%	 (original draft)}% \& review/editing)}
\end{list}

\end{document}
